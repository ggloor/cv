% XeLaTeX can use any Mac OS X font. See the setromanfont command below.
% Input to XeLaTeX is full Unicode, so Unicode characters can be typed directly into the source.

% The next lines tell TeXShop to typeset with xelatex, and to open and save the source with Unicode encoding.

%!TEX TS-program = xelatex
%!TEX encoding = UTF-8 Unicode

\documentclass[11pt]{article}

% See geometry.pdf to learn the layout options. There are lots.
\usepackage[papersize={8.5in,11in}, total={7in,9.5in},top=20mm, left=20mm, includefoot]{geometry}

\setlength{\parindent}{10pt}
\setlength{\parskip}{2pt}


\usepackage{graphicx}
\usepackage{amssymb}
\usepackage{etaremune}
%\bibliographystyle{natbib} % this must remain commented

\makeatletter
\long\def\thebibliography#1{%
  \section*{\refname}%
  \@mkboth{\MakeUppercase\refname}{\MakeUppercase\refname}
  \settowidth{\dimen0}{\@biblabel{#1}}%
  \setlength{\dimen2}{\dimen0}%
  \addtolength{\dimen2}{\labelsep}
  \sloppy
  \clubpenalty 4000 
  \@clubpenalty 
  \clubpenalty 
  \widowpenalty 4000
  \sfcode `\.\@m

  \renewcommand{\labelenumi}{\@biblabel{\theenumi}} % labels like [3], [2], [1]
  \begin{etaremune}[labelwidth=\dimen0,leftmargin=\dimen2]\@openbib@code
}
\def\endthebibliography{\end{etaremune}}
\def\@bibitem#1{%
  \item \if@filesw\immediate\write\@auxout{\string\bibcite{#1}{\the\value{enumi}}}\fi\ignorespaces
}
\makeatother
%\renewcommand\bibnumfmt[1]{\emph{\textbf{\small{#1)}}}}
%\usepackage{graphicx}[pdftex]
%\usepackage{wrapfig}
\usepackage{caption}
\linespread{1}
%\usepackage{setspace}
% Will Robertson's fontspec.sty can be used to simplify font choices.
% To experiment, open /Applications/Font Book to examine the fonts provided on Mac OS X,
% and change "Hoefler Text" to any of these choices.
\usepackage{xspace}
\usepackage{fontspec,xltxtra,xunicode}
\defaultfontfeatures{Mapping=tex-text}
\setromanfont[Mapping=tex-text]{Palatino}
\setsansfont[Scale=MatchLowercase,Mapping=tex-text]{Gill Sans}
\setmonofont[Scale=MatchLowercase]{Andale Mono}

%set the bibliography style
%\usepackage[super,sort&compress,comma]{natbib}
%%\bibliographystyle{unsrt}


%set up a header
\usepackage{fancyhdr}
\pagestyle{fancy}
\lhead{\textbf{GB Gloor}} \chead{\textbf{Curriculum Vitae}} \rhead{\textbf{ \today}}
\newcommand{\HRule}{\rule{\linewidth}{.8mm}}
\setlength{\headheight}{15pt}
\makeatletter
\renewcommand\section{\@startsection
	{section}{1}{0in}% %name level indent
	{0.1\baselineskip}%
	{0\baselineskip}%
	{\sffamily\bfseries\large}
	%
}
\makeatother

\makeatletter
\renewcommand\subsection{\@startsection
	{subsection}{2}{0.25in}% %name level indent
	{0.1\baselineskip}%
	{0\baselineskip}%
	{\textbf}%
}
\makeatother

\makeatletter
\renewcommand\subsubsection{\@startsection
	{subsubsection}{3}{0.25in}% %name level indent
	{0.1\baselineskip}%
	{0\baselineskip}%
	{\textit}%
}
\makeatother

\begin{document}
\begin{center}
\textbf{Gregory Gloor, PhD}

Professor, Department of Biochemistry

Schulich School of Medicine and Dentistry

Western University

Tel: (519) 661-3526; email: ggloor@uwo.ca 
 
http://ggloor.github.io
\end{center}
\section{Expertise and Research Interests}
\begin{description}\itemsep=2pt
\item Composition and function of the human and other microbiomes. I use and develop tools to examine 16S rRNA gene composition, gene expression of mixed population samples, and metabolomic analysis of clinical samples. I teach a graduate course on the use of compositional data analysis techniques to examine transcriptomes, microbiomes and other types of complex data sets derived from high throughput sequencing. 
\item Protein evolution. We use and develop tools to examine how protein structure and function is maintained in response to sequences changes. We have a special interest in identifying the role that variable positions play in protein evolution. I teach an undergraduate course in protein sequence alignment and proteins sequence-structure alignment. 
\item Computational biology and that application of techniques for compositional data analysis to the above problems. Our primary contributions so far have been the ALDEx2 tool in Bioconductor for the analysis of high throughput experiments that generate counts per sequence tag: 16S rRNA gene sequencing, transcriptomics and selex-type experiments. I have further tools under development, and have contributed new visualization methods (effect-size plots) to the field.
\end{description}

\section{Education and Training}
\begin{description}\itemsep=2pt
\item 1988-1990  Postdoctoral Fellow.  University of Wisconsin - Madison - Laboratory of Genetics. Supervisor: Dr. William Engels.
\item 1988  Ph.D, University of Western Ontario, Department of Biochemistry. Supervisor: Dr. George Chaconas. Dissertation: \textit{Characterization of the Integrative Precursor  Protein-DNA Complex of Bacteriophage Mu.}
\item 1983  HBSc, University of Western Ontario, Genetics
\end{description}

\section{Employment}\itemsep=2pt
\begin{description}\itemsep=2pt
\item[2002-present], Professor of Biochemistry\\ University of Western Ontario, Faculty of Medicine\\ now the Schulich School of Medicine and Dentistry
\item 1997-2002, Associate Professor of Biochemistry\\ University of Western Ontario, Faculty of Medicine
\item 1993-1997,	Assistant Professor of Biochemistry\\ University of Western Ontario, Faculty of Medicine
\item 1990-1992,	Assistant Professor Medical Genetics\\ Memorial University of Newfoundland, Faculty of Medicine
\end{description}
\section{Faculty Development}
\begin{description}
\item 2014, Five day theory and applied course on Compositional Data Analysis (UdG, Spain)\\ Accredited by European Statistical Society

\item 2007, Leadership Workshop\\ Offered by Continuing Education of Shulich School of Medicine and Dentistry for Medical School Accreditation Leaders
\item 1995,  Course on Teaching at the University 	Level\\ Offered through the Faculty of Medicine Development Office.
\item 1990, Faculty Orientation Day Memorial University of Newfoundland\\
		This covered teaching tips and grant writing skills for 	new faculty at MUN

\end{description}
\section{Awards, Honours, Fellowships}
\begin{description}\itemsep=2pt
\item 2014,       Faculty Development Award: Attended week-long course on Compositional Data Analysis (UdG, Spain)
\item 2011-2013,       Faculty Scholar
\item 2009,		University Student's Council Teaching Honor Roll
\item 2007,		University Student's Council Teaching Honor Roll
\item 2005,		Schulich School of Medicine Teaching Award
\item 2004,		WL Magee Teaching Award, Biochemistry, UWO
\item 1993 - 1998,	Salary Award, Medical Research Council of Canada (MRC)\\ Development Grant in Molecular Biology
\item 1984 -1988, 	K. M. Hunter Fellowship, National Cancer Institute of Canada.
\item 1983,	       	Graduate Entrance Scholarship, UWO.
\end{description}

\section{HQP Training Summary}

\begin{description}\itemsep=2pt
\item Graduate Student: 10;  Undergraduate Student: 30; Postdoctoral Fellow: 2; 
\item Graduate Advisory Committee: 30; Thesis Defence: 68; Qualifying Examiner: 46.
\end{description}

\section{Scholarly and Professional Activities Summary}

\subsection{Grants and Awards Panels, Editorial membership}

\begin{description}\itemsep=2pt
\item 2017-present, Chair, CIHR Project Grant Scheme (Genomics/Genetics)
\item 2017-present, Member, CIHR College of Reviewers (First round invitee)
\item 2017, Member, Canadian Crohns and Colitis Review panel
\item 2016, Member, Agence Nationale de la Recherche, Preindustrial Biotechnology Demonstrator, Paris, France
\item 2016-present, Senior Editor, Microbiome
\item 2016, Western Science and Engineering Review Board Member
\item 2016, CIHR Operating Grant Review Panel Chair
\item 2015-2016, Associate Editor, Microbiome
\item 2015, Ontario Genomics Institute: SPARC and Genome Canada review panel
\item 2014-present, CRC College of Reviewers
\item 2012-2015,	Editorial Board member Microbiome
\item 2010-2014,	Member CIHR Genetics panel 
\item 2008-present,	IODE Doctoral Scholarship committee 
\item 2006-2010, 	NCIC Model Organisms Panel B2 
\item 1998, 1999, 2000, 2003, 2004, 2005, 2006, 2007, 2008 MRC/CIHR BMB/Genetics/Genomics invitee
\\item 1999, 		Chair OGS Biochemistry/Biophysics panel 
item 1997-1999,  	OGS Biochemistry/Biophysics panel 
\item 1997–2000, 	NCIC Virology and Molecular Biology Committee\\ Panel F
\item  1995-2001, Foundation for Gene and Cell Therapy (Jesse's Journey)\\ Chair and Review Organizer 1995 - 2001
\end{description}

\subsection{Board memberships}
\begin{description}
\item 2001-2002,	Member, Board of Directors for Partners in Research
\item 1995-2001	Member, Board of Directors, The Foundation for Gene 	and Cell Therapy (FFGCT).\\ I was the lead negotiator for 	the FFGCT in working out a partnership with the MRC 	to fund up to 9 post-doctoral fellows in the area of gene 	therapy. This partnership provided approximately 	\$900,000 in new money to the MRC.
\item 1995,	Member of the Home Team for Jesse’s Journey 	(Internet and Science Advisor).\\ I posted and updated a 	World Wide Web map page so that people from around the 	world could follow John and Jesse’s progress.
\end{description}

\subsection{Presentations and Invitations}
\begin{description}\itemsep=2pt
\item 2017,  Keynote, Microbial Ecology 2017, Toronto, Ontario 
\item 2017,  Workshop, Compositional Data analysis methods, Microbial Ecology 2017, Toronto, Ontario 
\item 2017,  Invited speaker, EMBL-EBI Industrial Program Workshop - The human microbiome: challenges and opportunities for novel therapeutics, Hinxton, England
\item 2017,  Invited speaker Canadian Society of Microbiology, Waterloo, Ontario 
\item 2017,  Canadian Statistical Sciences Institute Microbiome Planning Meeting speaker and discussion leader, Winnipeg,Manitoba
\item 2017,  Contributed Oral Presentation (2), Great Lakes Bioinformatics, Chicago, Illinois
\item 2017,	 Invited speaker in the Microbiology \& Immunology Department, Western University, London,  CA
\item 2017,	 Invited speaker in the Health Sciences Department, Carleton University, Ottawa,  CA
\item 2016,	 Invited speaker in the Biostatistics and Epidemiology Department, Western University, London,  CA
\item 2016,  Invited speaker at Exploring Human Host-Microbiome Interactions in Health and Disease 2016, Cambridge, UK
\item 2016,  Invited workshop organizer at Exploring Human Host-Microbiome Interactions in Health and Disease 2016, Cambridge, UK 
\item 2016,	 Invited speaker at Symposium on Synthetic Biology, Western University, London,  CA
\item 2016,  Invited workshop presenter, The Human Microbiome and Epidemiology, 2016 Epidemiology Congress of the Americas, Miami, USA
\item 2016,  Invited presentation/workshop, Infection, Inflammation and Immunity course, The Arctic University of Norway, Tromso, NO
\item 2016,  Oral Presentation, Great Lakes Bioinformatics/Canadian Computational Biology Conference, Toronto, CA
\item 2015,  Invited speaker at Exploring Human Host-Microbiome Interactions in Health and Disease 2015, Cambridge, UK
\item 2015,  Invited paper at CoDaWork 2015, Girona, Spain
\item 2015,  Applying compositional data framework to microbiome datasets,  Canadian Society of Microbiology workshop 2015, Saskatoon, Canada
\item 2015,  Invited speaker, University of Guelph Bioinformatics group
\item 2014,     Invited seminar, Dept. of Biochemistry, University of Calgary
\item 2014, 	Invited participant at NIH sponsored Microbiome Quality Control Initiative: only Canadian group invited, Rockville, MD, USA
\item 2013, 	Invited speaker at Fondation Merieux Conference on Better Foods for Better Health, Annecy, France
\item 2013, 	Invited speaker at the Institute of Genome Sciences seminar series, University of Maryland, Baltimore, USA
\item 2013, 	Invited expert participant at African International Conference and Workshop on the Microbiome and Probiotics, Nairobi, Kenya
\item 2011, 	Invited speaker at the RePOOPulating the gut: therapeutic microbial preparations to eradicate recurrent C.difficile infections in Canada, Toronto
\item 2011, 	Invited expert participant at International Society for the Application of Probiotics and Prebiotics, Berkley, CA
\item 2010, 	Invited platform speaker at the Ontario Illumina Users Group, Toronto Ontario 
\item 2008, 	Invited speaker at University of North Carolina-Charlotte Department of Bioinformatics Seminar Series
\item 2003, Invited speaker at First Canadian Workshop on Statistical Genomics, The Fields Institute, Toronto 
\item 2000,	 Department of Genetics seminar series, Harvard Medical School, Boston, Mass.
\item 1999,	 London and Regional Cancer Center seminar series, London Ontario
\item  1997	Department of Molecular Biology and Genetics seminar series, 	University of Guelph
\item 1994	 Mogenson Research Forum, UWO 	Faculty of Medicine
\item 1992	Department of Genetics, University of 	Alberta seminar series\\
		Alberta Heritage Foundation for Medical Research 	Sponsored Speaker
\item  1992	Lunchtime Seminar Series, Faculty of Medicine, 	Memorial University of Newfoundland
\item 1992 Department of Biochemistry seminar series, Queens University
\item  1992 Department of Biochemistry seminar series, UWO
\item 1991 Molecular Biology Research Discussion Group seminar series, Faculty of 	Medicine, MUN
\item 1991	 Department of Biochemistry seminar series, MUN
\item 1990 Faculty of Medicine seminar series, MUN
\end{description}
%\renewcommand{\tab}{\hspace{1.5in}}


%\renewcommand{\tab}{\hspace{0.25in}}
\renewcommand\refname{}
\clearpage
\subsection{Peer Reviewed Papers:\\}

\subsection*{Academic trivia} 

\begin{itemize}
\item H-index: 34, i10 index: 75 (Google Scholar)
\item Erdos number 3 (two ways)
\item Aitchison number 1 (two ways)
\item Academic lineage: T.H. Morgan
\end{itemize} 

\nocite{gloorFrontiers:2017,bian:2017,Martz2017,McMillan2016,Ettinger:2017aa,Al:2017aa,Wolfs:2016aa,Slade:2016ab,Rahat-Rozenbloom:2016aa,Slade:2016aa,Petrova:2016aa,gloorAJS:2016,Gloor:2016cjm,Urbaniak:2016ac,gloor2016s,Gloor:2015,Wong:2016aa,Urbaniak:2016aa,Asemaninejad:2016aa,McMillan:2016aa,Walton:2016aa,Bisanz:2016aa,Bisanz:2015aa,St-Denis:2015aa,Goneau:2015ab,McMillan:2015aa,Martz:2015aa,Macklaim:2015aa,Yang:2015aa,Rahat-Rozenbloom:2014ab,mcmurrough:2014,Urbaniak:2014ab,Gan:2014aa,Reid:2014aa,Rahat-Rozenbloom:2014aa,Brace:2014,Bisanz:2014aa,kernohan:2014,Dickson:2014aa,Urbaniak:2014aa,Rosenthal:2014,Bisanz:2014ab,fernandes:2014,Di-Bella:2013aa,DaSilva:2013aa,fernandes:2013,Kim:2013aa,MacPhee:2013aa,Petrof:2013aa,Lahiry:2013aa,macklaim:2013,Anukam:2013aa,Burton:2013,Allen-Vercoe:2012,Allen-Vercoe:2012a,Macklaim:2012,Li:2012aa,Genereaux:2012,Kvas:2012,Dickson:2012,Turowec:2011a,Takeuchi:2011a,Macklaim:2011,Duncan:2011,Reid:2011a,Hummelen:2011,Hoke:2010a,Dickson:2010,Gloor:2010a,Fernandes:2010b,Kleinstiver:2010,Duncan:2010,Hummelen:2010,Fernandes:2010a,Gloor:2010,Lahiry:2009,Dunn:2008,Holmes:2006,Gloor:2005,Martin:2005,Dempsey:2004,Qin:2004,Gloor:2004,Coveny:2002,Gloor:2002,Gloor:2001,kari2001computer,Krishna:2001,kari2000using,Gloor:2000,Bassi:2000,daley1999circular,Gloor:1999,gloor1999towards,kari1999compute,Lankenau:1998,Gloor:1998,Dray:1997,Keeler:1997,Keeler:1996,Andrews:1995,Nassif:1994,Gloor:1993,Gloor:1991,Gloor:1988,Gloor:1986,Chaconas:1985a,Chaconas:1985,Faust:1984a,Faust:1984,Chaconas:1984aa}
\bibliographystyle{unsrt}
\vspace{2pt}\bibliography{bibdesk_refs}

\subsection{Non Peer Reviewed Manuscripts}\  \\

Russell J Dickson and Gregory B Gloor. Xorro: Rapid paired-end read overlapper. arXiv preprint arXiv:1304.4620, 2013.

Russell J Dickson and Gregory B Gloor.The MIp  toolset:an efficient algorithm for calculating mutual information in protein alignments. arXiv preprint arXiv:1304.4573, 2013.\\

\subsection{Software releases}\ \\

ALDEx2. ALDEx tool to examine compositional high-throughput sequence data with Welch's t-test and Wilcoxon rank test. https://github.com/ggloor/ALDEx2, and\\ http://www.bioconductor.org/packages/release/bioc/html/ALDEx2.html last update Oct 2017\\

Languages and utilities: R, bash, Perl,  awk, \LaTeX, Markdown, HTML, git, svn

\newpage
\section{ Research Funding History}


%%%
\begin{description}
\setlength\itemsep{0em}

\item[NSERC Discovery, 2015-2020:] Molecular covariation in protein families

\setlength\itemindent{-1em}

\item {\em PI: Gloor, GB}
\item Goal is to examine how  covarying positions affect the structure and function of protein families that can be used as gene editing reagents. 	
\item total: \$155000 - pays for student and supplies

\end{description}
%%%%%%
\begin{description}
\setlength\itemsep{0em}

\item[NIH R21 Dec 2015-2017, NIH R33 2017-2018:] Microbes that matter

\setlength\itemindent{-1em}

\item {\em PI: Elaine Petrof (Queen's U)}, Allen-Vercoe (Guelph), Gloor (UWO)
\item Developing a synthetic stool substitute for the treatment of recurrent \emph{C. difficile} infection	
\item direct to Gloor: \$70000 - partially pays for one student and sequencing costs

\end{description}
%%%%%%
\begin{description}
\setlength\itemsep{0em}

\item[CIHR 2013-2016:] Role of intestinal microbiota in non-alcoholic fatty liver disease pre and post bariatric surgery

\setlength\itemindent{-1em}

\item {\em PI: ALLARD, Johane P (U. Toronto): }, Comelli Elena M; GLOOR, Gregory B; JACKSON, Timothy D; LOU, Wen-Yi W; OKRAINEC, Allan
\item Characterization of the stool microbiota in a cohort of patients undergoing treatment for non-alcoholic fatty liver disease	
\item total: 522169, direct to Gloor: \$25000/yr - partially paid for one student and sequencing costs
\end{description}
%%%%%%
\begin{description}
\setlength\itemsep{0em}

\item[CIHR 2014-1016:] Intestinal microbiome and extremes of atherosclerosis

\setlength\itemindent{-1em}

\item {\em PI: SPENCE, J. David },  ALLEN-VERCOE, Emma; GLOOR, Gregory B; REID, Gregor
\item Characterization of the stool microbiota in a cohort of patients screened for risk of atherosclerosis	
	
\item total: 211600, direct to Gloor: \$25000 - partially paid for one student and sequencing costs
\end{description}
%%%%%%
\begin{description}
\setlength\itemsep{0em}

\item[CIHR 2013-1016:] Non-Alcoholic Fatty Liver Disease: Role of Intestinal Microbiota\\ and n-3 Polyunsaturated Fatty Acid Supplementationtle

\setlength\itemindent{-1em}

\item {\em PI: ALLARD, Johane P  },  COMELLI, Elena M; GLOOR, Gregory B; LOU, Wen-Yi W
\item Characterization of the stool microbiota in a cohort of patients treated for non-alcoholic fatty liver disease with fish oils	
\item total 211600, direct to Gloor: \$17000 - partially paid for one student
\end{description}
%%%%%%
\begin{description}
\setlength\itemsep{0em}

\item[CIHR Team grant, 2010-2015:] The Vaginal Microbiome Project Team

\setlength\itemindent{-1em}

\item {\em PI: MONEY, Deborah M}, BOCKING, Alan D; HEMMINGSEN, Sean M; HILL, Janet E; REID, Gregor (Co-PIs), co-investigators: DUMONCEAUX, Timothy J; GLOOR, Gregory B; LINKS, Matthew G; O'DOHERTY, Kieran C; TANG, Patrick K; VAN SCHALKWYK, Julianne E; YUDIN, Mark H
\item Collection and analysis of large vaginal microbiota cohorts to identify determinants of health and disease in the Canadian population	
\item total:1745341, direct to Gloor: 15000/year - partially paid for one student. My role was tool development

\end{description}
%%%%%%
\begin{description}
\setlength\itemsep{0em}

\newpage
\item[Southeastern Ontario Academic Medical Organization 2014-2015:] Elucidating the factors that determine success in fecal transplant therapy for {\em C. difficile} infection

\setlength\itemindent{-1em}

\item {\em PI: PETROF, E}, ROPELSKI, Mark, ALLEN-VERCOE, Emma, GLOOR, Gregory
\item Identifying mechanisms of microbial ecosystem inhibition of {\em C. difficile}
\item total:92000, direct to Gloor: \%14000 - paid for sequencing costs

\end{description}
%%%%%%
\begin{description}
\setlength\itemsep{0em}

\item[CIHR Network grant 2012-2015] Maternal-Infant Microbiome and Immunity (MIMI) Network

\setlength\itemindent{-1em}

\item {\em PI: KOLLMANN, Tobias R}, GLOOR, Gregory B; REID, Gregor
\item Team grant to further training and planning of microbiome effects on proper health and development in Africa.	
\item total:600000, direct to Gloor: 200000 - paid for PDF, student and conference costs

\end{description}
%%%%%%
\begin{description}
\setlength\itemsep{0em}

\item[NSERC ENGAGE, 2014-2015:] Function of maize endophytic microbiome

\setlength\itemindent{-1em}

\item {\em PI: GLOOR, GREGORY } with A\&L Biologicals
	
\item Total: \$25000 - sequencing and sample processing costs
\item Determine the functional differences between high and normal yield sites by examining the endophytic corn sap meta-transcriptome.
\end{description}
%%%%%%
\begin{description}
\setlength\itemsep{0em}

\item[Ontario Centre of Excellence, 2014-2015:] Meta-transcriptome of high-yield corn endophytic microbiome

\setlength\itemindent{-1em}

\item {\em PI: GLOOR, GREGORY}, with A\&L Biologicals
\item Determine the functional differences between high and normal yield sites by examining the endophytic corn sap meta-transcriptome.	
\item Total: \$25000 - sequencing and sample processing costs

\end{description}
%%%%%%
\begin{description}
\setlength\itemsep{0em}

\item[CIHR Team grant in Bariatric Care, 2014-2019:] Exploiting the therapeutic effects of the fecal microbiome in bariatric care

\setlength\itemindent{-1em}

\item {\em PI: ALLARD, Johane P}, GAISANO, Herbert Y , BANKS, Kate; COMELLI, Elena M; GLOOR, Gregory B; HOTA, Susy S; JACKSON, Timothy D; LOU, Wen- Yi W; OKRAINEC, Allan; PHILPOTT, Dana J; POUTANEN, Susan M 
\item Fecal microbiome transplant as a treatment for obesity	
\item direct to Gloor: \$0 - I have chosen to withdraw from this research team 

\end{description}
%%%%%%
\begin{description}
\setlength\itemsep{0em}

\item[Agriculture and Agrifoods Canada, Agricultural Innovation Program, 2015-2017:] Developing molecular methods as diagnostic tools to identify biological factors contributing to crop productivity and soil health

\setlength\itemindent{-1em}

\item {\em PI: Dr. George Lazarovitz (A\&L Biologicals)}, Gloor, Gregory
\item Monitoring and understanding corn endophytic communities to maximize crop yield	
\item Total: \$600000, direct to Gloor: \$120000 - pays for one PDF and sequencing costs

\end{description}
%%%%%%
\begin{description}
\setlength\itemsep{0em}

\item[Academic Development Fund UWO, 2012:] Major request for Illumina MiSeq Instrument

\setlength\itemindent{-1em}

\item {\em PI: Gloor}, Hegele, Edgell, Singh
\item ADF money to buy an Illumina MiSeq for the Robarts sequencing core facility	
\item Total: \$132500 - Paid for purchase and installation.

\end{description}
%%%%%%
\begin{description}
\setlength\itemsep{0em}

\item[Ontario Genomics Institute Summer Research, 2010:] Metagenomic error rate analysis of rare biomes

\setlength\itemindent{-1em}

\item {\em PI: Gloor}
\item Summer research project examining metagenomic sequencing error rates	
\item Total: \$5000 - partially paid for one summer student
\end{description}
%%%%%%
\begin{description}
\setlength\itemsep{0em}

\item[NSERC Discovery, 2010-2015:] Molecular covariation in protein families

\setlength\itemindent{-1em}

\item {\em PI: Gloor}
	
\item Total: \$255564 - Core Research Funding

\end{description}
%%%%%%
\begin{description}
\setlength\itemsep{0em}

\item[Academic Development fund, UWO, 2010:] Deep resequencing of single genes

\setlength\itemindent{-1em}

\item {\em PI: Gloor}
\item Developing methodologies for resequencing on the Illumina MiSeq Sequencing Platform	
\item direct to Gloor: \$7500 - partially pays for one student and sequencing costs

\end{description}
%%%%%%
\begin{description}
\setlength\itemsep{0em}

\item[CIHR Operating, 2006-2010:] MODULATORS OF DOUBLE-STRAND BREAK REPAIR IN DROSOPHILA

\setlength\itemindent{-1em}

\item {\em PI: Gloor}
\item Core research funding for examining DSB repair in Drosophila somatic cells	
\item Total: \$405312 - Core Lab Research funding

\end{description}
%%%%%%
\begin{description}
\setlength\itemsep{0em}

\item[Academic Development Fund, UWO, 2008:] IDENTIFYING AND VALIDATING COEVOLVING POSITIONS

\setlength\itemindent{-1em}

\item {\em PI: Gloor}, Wahl, Dunn
	
\item direct to Gloor: \$8500 - partially paid for one PDF

\end{description}
%%%%%%
\begin{description}
\setlength\itemsep{0em}

\item[CANCER RESEARCH SOCIETY, 1997-1999:] GENES INVOLVED IN SOMATIC-CELL DOUBLE-STRAND REPAIR

\setlength\itemindent{-1em}

\item {\em PI: Gloor}
\item Core research lab funding	
\item Total: \$85145 - partially pays for one student and sequencing costs

\end{description}
%%%%%%
\begin{description}
\setlength\itemsep{0em}

\item[CIHR Operating, 2005-2008:] MODULATORS OF DOUBLE-STRAND BREAK REPAIR IN DROSOPHILA

\setlength\itemindent{-1em}

\item {\em PI: Gloor} 
\item Core research lab funding	
\item Total: \$361580 - partially pays for one student and sequencing costs

\end{description}
%%%%%%
\begin{description}
\setlength\itemsep{0em}

\item[CIHR Operating, 2000-2006:] Double strand break repair in Drosophila somatic and germline cells

\setlength\itemindent{-1em}

\item {\em PI: Gloor}
\item Core research lab funding		
\item Total: \$148000 - Core Lab research funding

\end{description}
%%%

%%%
\begin{description}
\setlength\itemsep{0em}

\item[MRC of Canada, 1997-2000:] TEMPLATE-DEPENDENT REPAIR OF DOUBLE STRAND CHROMOSOME BREAKS IN Drosophila

\setlength\itemindent{-1em}

\item {\em PI: Gloor}
	
\item Total: \$105005 -  Core Lab research funding

\end{description}
%%%%%%
\begin{description}
\setlength\itemsep{0em}

\item[MRC of Canada, 1993-1998:] Transposon induced gene targeting

\setlength\itemindent{-1em}

\item {\em PI: Gloor}
	
\item direct to Gloor: \$156920 -  Core Lab research funding
\end{description}
%%%

%%%
\begin{description}
\setlength\itemsep{0em}

\item[Cancer Research Society, 1992-1994:] Gene targeting to arbitrary sites in the genome

\setlength\itemindent{-1em}

\item {\em PI: Gloor}
	
\item Total: \$96000 - Core Research Lab Funding

\end{description}
%%%%%%
\begin{description}
\setlength\itemsep{0em}

\item[Where, years:] Title

\setlength\itemindent{-1em}

\item {\em PI: }, co-apps
	
\item Total: \$70000 - Core Research Lab Funding


\end{description}
%%%%%%
\begin{description}
\setlength\itemsep{0em}

\item[MRC of Canada, 1991-1994:] P element regulation and double strand break repair in Drosophila

\setlength\itemindent{-1em}

\item {\em PI: Gloor }
	
\item Total: \$197806 - Core Research Lab Funding

\end{description}
%%%



\end{document}
